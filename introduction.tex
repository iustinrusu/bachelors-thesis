\chapter*{Introduction}

\pagestyle{fancy}

\addcontentsline{toc}{chapter}{Introduction}

\markboth{INTRODUCTION}{}

\label{introduction}

Today, the digital entertainment industry is rapidly becoming the largest and highest earning in the world, threatening the old, well-established industries. More and more people prefer to play video games or enjoy augmented reality (AR) or virtual reality (VR) content than ever. For example, almost 1 in 2 Germans are gamers, out of which the vast majority plays regularly\cite{germans}. The number of related jobs is also increasing. A higher number of people that enjoy this domain now consider these jobs safe and viable, which in turn further develops the industry and improves the quantity, but also the quality of the products. Obviously, people also have to be trained, so schools and universities have a much wider range of options when it comes to game development, design and art.\\

There are many methods of creating digital content. Through the internet, anyone can access software and learning materials in order to create anything from textures and concept art, to 3D models and even fully playable games. Manually creating the assets needed for a game or any other application was and still is the main method, but sometimes, for some projects, doing all the work by hand is just impossible. Procedural generation, the key aspect discussed in this thesis, is a method of generating content algorithmically, without much, or even any human input. Developers can write functions that make use of pseudo-randomness to generate potentially infinite worlds, among other things such as textures and geometrical shapes\cite{procgenwiki}.\\

The main goal of this thesis is presenting an original take on a system that infinitely generates believable virtual terrain, mainly using Perlin noise, an algorithm of great importance in the world of procedural generation, covered in the third chapter, entitled \textit{Proposed approach}. I will also talk about some other noise algorithms and their uses, and some clever ways of optimizing the generation speed. When the complete system is put together, due to the high number of computations, the amount of vertices that have to be drawn on the user's screen and also other factors, the application might run slow on older hardware, so the need of optimization is great. Popular simulations of real-world phenomena that enhance the virtual environments, such as hydraulic erosion and volumetric clouds will also be discussed in the \textit{Theoretical aspects} chapter.