\chapter{Conclusion}

\pagestyle{fancy}

\label{conclusion}

While the proposed solution successfully achieves its goals and offers basic support for infinite procedurally generated terrain, it certainly lacks some more advanced features, that tools made for this specific purpose might have. Firstly, support for more complex algorithms for landscaping are of utmost importance, and should be on top of the priorities list. The application should be ready to output terrain with more realistic details, if the user requests it. Another important aspect is further optimization. If used in the context of a video game, where a lot of other factors are taken into consideration, the performance might drop in intense situations, where the engine struggles. Areas that need more attention are \textit{collision map} generation speed and overall \textit{code refactorization}. While trying to output as many frames per second as possible, another thing to take into consideration is moving more of the intense computations on separate threads, splitting the workload. As final touches, a better movement controller should be implemented, allowing the player to control an airplane, for example. Following this idea, in the current state of the application, a \textit{flight simulator} type of game can be easily set up: a simple 3D model can be designed in a software such as \textit{Blender}, or imported from the asset store, and the simulation of \textit{volumetric clouds} would greatly help with improving the overall atmosphere and immersion.\\

All in all, I believe this thesis serves as a great compendium of methods and algorithms that solve a multitude of procedural generation problems, and \textbf{\textit{terra}} is, in my opinion, a solid foundation for others to build on, and further expand.